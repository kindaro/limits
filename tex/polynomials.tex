\section{ Polynomials. }

\subsection{ Notation. }

Polynomials are given in two isomorphic shapes, for which we use a shorthand notation.  A sum is
written thus:
$$ \sum_{0}^{p} a_i x^{p - i} = \f{p}{a_0, a_1, \dots, a_m} $$
--- And a product thus:
$$ a_0\prod_{1}^{p} (x - x_i) = \poly {a_0} { x_0^{p_1}, \dots, x_n^{p_n} } $$
\itbr{Here, $n$ --- number of unique roots, $p = \sum{}{} p_i$ --- degree of the polynomial in
question, $p_i$ --- multiplicity of $i$th root.}

Though isomorphic, these forms are not on par with regard to information content: from a
$\poly{}{\dots}$ we may always obtain the corresponding $\f{}{\dots}$, but not necessarily the
other way around.

\subsection{ Ring operations. }

\begin{description}

    \item [Addition] You can sum polynomials as usual numbers in a positional number system
        \itbr{with the power of the variable being the position} except that you never carry. It
        is an unfortunate fact that $\poly{}{}$ has to be converted to $\f{}{}$ in the process and
        the information about roots lost --- the same as prime factorization is lost with the
        usual addition of integers.

        The degree of the resulting polynomial is the maximum of the degrees of the summands,
        unless a run of coefficients from $a_0$ to $a_q$ happen to each sum to zero, in which case
        it is reduced by $q + 1$.

    \item [Additive inverse] You can invert any polynomial by inverting every single value
        throughout $\f{}{}$ or by inverting $a_0$ in $\poly{a_0}{}$.
        
        The degree is invariant with regard to this operation.

    \item [Multiplication] The union of the multisets of the roots given by $\poly{}{}$ is the
        fast way to obtain the product of the corresponding polynomials. The usual positional
        multiplication, performed without carry, as in addition explained above, takes more
        effort.

        The degree of the resulting polynomial is the sum of the degrees of the multiples.

\end{description}
